% vim:encoding=utf8 ft=tex sts=2 sw=2 et:

\documentclass{classrep}
\usepackage[utf8]{inputenc}

\usepackage{listings}
\usepackage{graphicx}
\usepackage[T1]{fontenc}
\usepackage[variablett]{lmodern}
\studycycle{Informatyka, studia niestacjonarne}
\coursesemester{IV}

%\coursename{Angelologia teoretyczna i stosowana}
\coursename{Inteligentna Analiza Danych}
\courseyear{2017/2018}

\courseteacher{mgr Rogalski}
\coursegroup{sobota, 10:30}

\author{
  \studentinfo{Marcin Pajkowski}{211968} \and
  \studentinfo{Rafał Warda}{214067}
}

\title{Zadanie 2b: Klasyfikacja}

\begin{document}
\maketitle
\newpage
\section{Wstęp}
Celem zadania było wykorzystanie wcześniej napisanego programu implementującego koncepcję perceptronu wielowarstwowego do klasyfikacji zestawu "Iris Flower Data Set".

\section{Opis realizacji zadania}
\subsection{Odwzorowanie tekstowej reprezentacji danych}
Zestaw irysów składa się z 150 rekordów. W każdym wierszu pierwsze cztery wartości numeryczne opisują atrybuty kwiatu - kolejno długość i szerokość działki kielicha oraz długość i szerokość płatka. Wartości te są wyrażone w centymetrach. Piątym polem jest etykieta klasy:

- Iris-setosa,

- Iris-versicolor,

- Iris-virginica.
\newline
Aby etykiety klas przekształcić na wartość liczbową (i tym samym przystosować format danych do zgodnego z specyfiką projektu) wprowadzono wyjścia, które będą kolejno reprezentować stuprocentowe prawdopodobieństwo przynależności do danej klasy (oraz dla dwóch pozostałych - prawdopodobieństwo równe 0\%). Zgodnie z przyjętą nomenklaturą predykowane wyjście dla klasy "setosa" to 1 0 0, dla klasy "versicolor" 0 1 0. a dla klasy "virginica" 0 0 1. Odpowiednio spreprawowany plik tekstowy jest podawany do aplikacji.
\subsection{Podział zbioru}
Zbiór kwiatów jest dzielony na następujące zbiory: treningowy, testowy oraz walidacyjny - proporcja 6:3:1.
Przypisanie rekordów do odpowiednich zbiorów następuje w sposób losowy - z tego względu podana wyżej proporcja nie jest idealnie zachowana.
\subsection{Część badawcza}
\subsubsection{Opis części badawczej}
Manipulatory takie jak współczynnik momentum, współczynnik nauki czy obecność biasu pozostaną stałe dla całego badania, przyjmą następujące wartości:

-momentum: 0.9,

-nauka: 0.01,

-bias: aktywny.
\newline
Liczba epok: W pierwszej próbie 10000, w drugiej - 140000,
\newline
Częstotliwość sprawdzania błędu za pomocą zestawu testowego: 1 / 20 epok,
\newline
Kryterium udanej klasyfikacji: 99\% pewności.
\newline
Na listingach zaprezentowane są oczekiwane odpowiedzi i odpowiedzi sieci dla każdego kwiatu ze zbioru walidacyjnego.
\newpage
\subsection{Wyniki}
\subsubsection{Konfiguracja 4-4-3, 10000 epok}
\begin{lstlisting}[basicstyle=\small]
=== Attributes: 5.4 3.7 1.5 0.2
Expected: 1.000000, output: 0.987056
Expected: 0.000000, output: 0.015400
Expected: 0.000000, output: 0.000109
=== Attributes: 5.1 3.8 1.5 0.3 
Expected: 1.000000, output: 0.986976
Expected: 0.000000, output: 0.015493
Expected: 0.000000, output: 0.000109
=== Attributes: 4.5 2.3 1.3 0.3 
Expected: 1.000000, output: 0.984507
Expected: 0.000000, output: 0.018438
Expected: 0.000000, output: 0.000117
=== Attributes: 5.7 2.8 4.5 1.3 
Expected: 0.000000, output: 0.012390
Expected: 1.000000, output: 0.991941
Expected: 0.000000, output: 0.005329
=== Attributes: 5.8 2.7 4.1 1 
Expected: 0.000000, output: 0.012922
Expected: 1.000000, output: 0.992662
Expected: 0.000000, output: 0.004535
=== Attributes: 6.3 2.5 4.9 1.5 
Expected: 0.000000, output: 0.011666
Expected: 1.000000, output: 0.911234
Expected: 0.000000, output: 0.071384
=== Attributes: 6 2.9 4.5 1.5 
Expected: 0.000000, output: 0.012341
Expected: 1.000000, output: 0.990407
Expected: 0.000000, output: 0.006424
=== Attributes: 6.8 3 5.5 2.1 
Expected: 0.000000, output: 0.010184
Expected: 0.000000, output: 0.024265
Expected: 1.000000, output: 0.979700
=== Attributes: 6.4 3.2 5.3 2.3 
Expected: 0.000000, output: 0.010154
Expected: 0.000000, output: 0.021421
Expected: 1.000000, output: 0.982239
=== Attributes: 6.7 3.3 5.7 2.1 
Expected: 0.000000, output: 0.010213
Expected: 0.000000, output: 0.027401
Expected: 1.000000, output: 0.976872
=== Attributes: 6.4 2.8 5.6 2.1 
Expected: 0.000000, output: 0.010123
Expected: 0.000000, output: 0.018746
Expected: 1.000000, output: 0.984612



=== Attributes: 6.3 3.4 5.6 2.4 
Expected: 0.000000, output: 0.010120
Expected: 0.000000, output: 0.018502
Expected: 1.000000, output: 0.984826
\end{lstlisting}

Sieci w tej konfiguracji udało się sklasyfikować 3/12 okazów. Należy jednak zwrócić uwagę na to, że sieć była pewna każdej oczekiwanej odpowiedzi w ponad 90%

\subsubsection{Konfiguracja 4-4-3, 140000 epok}
\begin{lstlisting}[]
=== Attributes: 5.4 3.7 1.5 0.2 
Expected: 1.000000, output: 0.995840
Expected: 0.000000, output: 0.005799
Expected: 0.000000, output: 0.000007
=== Attributes: 5.4 3.9 1.3 0.4 
Expected: 1.000000, output: 0.995847
Expected: 0.000000, output: 0.005785
Expected: 0.000000, output: 0.000007
=== Attributes: 5 3 1.6 0.2 
Expected: 1.000000, output: 0.995714
Expected: 0.000000, output: 0.006027
Expected: 0.000000, output: 0.000007
=== Attributes: 5.2 3.5 1.5 0.2 
Expected: 1.000000, output: 0.995826
Expected: 0.000000, output: 0.005824
Expected: 0.000000, output: 0.000007
=== Attributes: 4.9 3.1 1.5 0.1 
Expected: 1.000000, output: 0.995785
Expected: 0.000000, output: 0.005897
Expected: 0.000000, output: 0.000007
=== Attributes: 5 3.3 1.4 0.2 
Expected: 1.000000, output: 0.995818
Expected: 0.000000, output: 0.005838
Expected: 0.000000, output: 0.000007
=== Attributes: 5.6 3 4.5 1.5 
Expected: 0.000000, output: 0.004205
Expected: 1.000000, output: 0.999816
Expected: 0.000000, output: 0.000018
=== Attributes: 6.3 2.5 4.9 1.5      <= nierozpoznany kwiat
Expected: 0.000000, output: 0.000010
Expected: 1.000000, output: 0.039244
Expected: 0.000000, output: 0.984793
=== Attributes: 6.7 3 5 1.7 
Expected: 0.000000, output: 0.002279
Expected: 1.000000, output: 0.999397
Expected: 0.000000, output: 0.000084


=== Attributes: 6.3 3.3 6 2.5 
Expected: 0.000000, output: 0.000003
Expected: 0.000000, output: 0.003929
Expected: 1.000000, output: 0.999221
=== Attributes: 7.7 2.8 6.7 2 
Expected: 0.000000, output: 0.000003
Expected: 0.000000, output: 0.003929
Expected: 1.000000, output: 0.999221
=== Attributes: 6.4 2.8 5.6 2.2 
Expected: 0.000000, output: 0.000003
Expected: 0.000000, output: 0.003929
Expected: 1.000000, output: 0.999221
=== Attributes: 6.1 2.6 5.6 1.4 
Expected: 0.000000, output: 0.000003
Expected: 0.000000, output: 0.003931
Expected: 1.000000, output: 0.999221
=== Attributes: 6.9 3.1 5.4 2.1 
Expected: 0.000000, output: 0.000003
Expected: 0.000000, output: 0.004255
Expected: 1.000000, output: 0.999137
\end{lstlisting}

Sieci w tej samej konfiguracji udało się sklasyfikować 13/14 okazów. Tym, co zasługuje w tej próbie na szczególną uwagę jest właśnie niesklasyfikowany przypadek - sieć oceniła wysoko możliwość przydziału do błędnej klasy.

\subsubsection{Konfiguracja 4-7-3, 10000 epok}
\begin{lstlisting}[basicstyle=\small]
=== Attributes: 5.4 3.9 1.3 0.4 
Expected: 1.000000, output: 0.991151
Expected: 0.000000, output: 0.011152
Expected: 0.000000, output: 0.000012
=== Attributes: 5.2 3.4 1.4 0.2 
Expected: 1.000000, output: 0.990839
Expected: 0.000000, output: 0.011379
Expected: 0.000000, output: 0.000012
=== Attributes: 6.9 3.1 4.9 1.5 
Expected: 0.000000, output: 0.011950
Expected: 1.000000, output: 0.997761
Expected: 0.000000, output: 0.001007
=== Attributes: 6.3 3.3 4.7 1.6 
Expected: 0.000000, output: 0.012261
Expected: 1.000000, output: 0.997566
Expected: 0.000000, output: 0.001118
=== Attributes: 6 2.2 4 1 
Expected: 0.000000, output: 0.012147
Expected: 1.000000, output: 0.997714
Expected: 0.000000, output: 0.000867



=== Attributes: 5.6 2.5 3.9 1.1 
Expected: 0.000000, output: 0.012488
Expected: 1.000000, output: 0.997670
Expected: 0.000000, output: 0.000868
=== Attributes: 6 2.9 4.5 1.5 
Expected: 0.000000, output: 0.012640
Expected: 1.000000, output: 0.996556
Expected: 0.000000, output: 0.001719
=== Attributes: 5.5 2.6 4.4 1.2 
Expected: 0.000000, output: 0.012176
Expected: 1.000000, output: 0.991186
Expected: 0.000000, output: 0.005115
=== Attributes: 6.2 2.9 4.3 1.3 
Expected: 0.000000, output: 0.012179
Expected: 1.000000, output: 0.997802
Expected: 0.000000, output: 0.000876
=== Attributes: 6.5 3 5.8 2.2 
Expected: 0.000000, output: 0.001779
Expected: 0.000000, output: 0.001134
Expected: 1.000000, output: 0.999491
=== Attributes: 5.8 2.8 5.1 2.4 
Expected: 0.000000, output: 0.001780
Expected: 0.000000, output: 0.001130
Expected: 1.000000, output: 0.999490
=== Attributes: 6.4 2.8 5.6 2.1 
Expected: 0.000000, output: 0.001773
Expected: 0.000000, output: 0.001138
Expected: 1.000000, output: 0.999484
=== Attributes: 7.4 2.8 6.1 1.9 
Expected: 0.000000, output: 0.001734
Expected: 0.000000, output: 0.001206
Expected: 1.000000, output: 0.999439
=== Attributes: 7.9 3.8 6.4 2 
Expected: 0.000000, output: 0.001455
Expected: 0.000000, output: 0.011966
Expected: 1.000000, output: 0.990354
=== Attributes: 7.7 3 6.1 2.3 
Expected: 0.000000, output: 0.001738
Expected: 0.000000, output: 0.001209
Expected: 1.000000, output: 0.999442
=== Attributes: 6.7 3.3 5.7 2.5 
Expected: 0.000000, output: 0.001765
Expected: 0.000000, output: 0.001162
Expected: 1.000000, output: 0.999475
\end{lstlisting}

Sieć rozpoznała wszystkie kwiaty w pierwszej próbie.
\newpage
\subsubsection{Konfiguracja 4-7-3, 140000 epok}
\begin{lstlisting}[basicstyle=\small]
=== Attributes: 5.4 3.7 1.5 0.2 
Expected: 1.000000, output: 0.995840
Expected: 0.000000, output: 0.005799
Expected: 0.000000, output: 0.000007
=== Attributes: 5.4 3.9 1.3 0.4 
Expected: 1.000000, output: 0.995847
Expected: 0.000000, output: 0.005785
Expected: 0.000000, output: 0.000007
=== Attributes: 5 3 1.6 0.2 
Expected: 1.000000, output: 0.995714
Expected: 0.000000, output: 0.006027
Expected: 0.000000, output: 0.000007
=== Attributes: 5.2 3.5 1.5 0.2 
Expected: 1.000000, output: 0.995826
Expected: 0.000000, output: 0.005824
Expected: 0.000000, output: 0.000007
=== Attributes: 4.9 3.1 1.5 0.1 
Expected: 1.000000, output: 0.995785
Expected: 0.000000, output: 0.005897
Expected: 0.000000, output: 0.000007
=== Attributes: 5 3.3 1.4 0.2 
Expected: 1.000000, output: 0.995818
Expected: 0.000000, output: 0.005838
Expected: 0.000000, output: 0.000007
=== Attributes: 5.6 3 4.5 1.5 
Expected: 0.000000, output: 0.004205
Expected: 1.000000, output: 0.999816
Expected: 0.000000, output: 0.000018
=== Attributes: 6.3 2.5 4.9 1.5      <= nierozpoznany kwiat
Expected: 0.000000, output: 0.000010
Expected: 1.000000, output: 0.039244
Expected: 0.000000, output: 0.984793
=== Attributes: 6.7 3 5 1.7 
Expected: 0.000000, output: 0.002279
Expected: 1.000000, output: 0.999397
Expected: 0.000000, output: 0.000084
=== Attributes: 6.3 3.3 6 2.5 
Expected: 0.000000, output: 0.000003
Expected: 0.000000, output: 0.003929
Expected: 1.000000, output: 0.999221
=== Attributes: 7.7 2.8 6.7 2 
Expected: 0.000000, output: 0.000003
Expected: 0.000000, output: 0.003929
Expected: 1.000000, output: 0.999221
=== Attributes: 6.4 2.8 5.6 2.2 
Expected: 0.000000, output: 0.000003
Expected: 0.000000, output: 0.003929
Expected: 1.000000, output: 0.999221

=== Attributes: 6.1 2.6 5.6 1.4 
Expected: 0.000000, output: 0.000003
Expected: 0.000000, output: 0.003931
Expected: 1.000000, output: 0.999221
=== Attributes: 6.9 3.1 5.4 2.1 
Expected: 0.000000, output: 0.000003
Expected: 0.000000, output: 0.004255
Expected: 1.000000, output: 0.999137
\end{lstlisting}

Błędem sieci w tej konfiguracji jest ten sam reprezentant populacji irysów co w sieci "4-3-3/140000".

\subsubsection{Konfiguracja 4-5-4-3, 10000 epok}
\begin{lstlisting}[basicstyle=\small]
=== Attributes: 5.7 4.4 1.5 0.4 
Expected: 1.000000, output: 0.991547
Expected: 0.000000, output: 0.014142
Expected: 0.000000, output: 0.000008
=== Attributes: 5 3.4 1.6 0.4 
Expected: 1.000000, output: 0.991542
Expected: 0.000000, output: 0.014193
Expected: 0.000000, output: 0.000009
=== Attributes: 4.7 3.2 1.6 0.2 
Expected: 1.000000, output: 0.991534
Expected: 0.000000, output: 0.014210
Expected: 0.000000, output: 0.000009
=== Attributes: 5.5 3.5 1.3 0.2 
Expected: 1.000000, output: 0.991837
Expected: 0.000000, output: 0.013771
Expected: 0.000000, output: 0.000008
=== Attributes: 5 3.5 1.3 0.3 
Expected: 1.000000, output: 0.991713
Expected: 0.000000, output: 0.013945
Expected: 0.000000, output: 0.000008
=== Attributes: 5.1 3.8 1.6 0.2 
Expected: 1.000000, output: 0.991468
Expected: 0.000000, output: 0.014281
Expected: 0.000000, output: 0.000008
=== Attributes: 4.9 2.4 3.3 1 
Expected: 0.000000, output: 0.191314
Expected: 1.000000, output: 0.804195
Expected: 0.000000, output: 0.000211
=== Attributes: 6.7 3 5 1.7 
Expected: 0.000000, output: 0.000214
Expected: 1.000000, output: 0.231048
Expected: 0.000000, output: 0.759830
=== Attributes: 5.7 2.6 3.5 1 
Expected: 0.000000, output: 0.070457
Expected: 1.000000, output: 0.921586
Expected: 0.000000, output: 0.000432


=== Attributes: 5.5 2.4 3.7 1 
Expected: 0.000000, output: 0.005127
Expected: 1.000000, output: 0.992765
Expected: 0.000000, output: 0.001950
=== Attributes: 5.7 3 4.2 1.2 
Expected: 0.000000, output: 0.047576
Expected: 1.000000, output: 0.943697
Expected: 0.000000, output: 0.000454
=== Attributes: 6.1 2.6 5.6 1.4 
Expected: 0.000000, output: 0.000077
Expected: 0.000000, output: 0.011492
Expected: 1.000000, output: 0.985483
=== Attributes: 6.4 3.1 5.5 1.8 
Expected: 0.000000, output: 0.000078
Expected: 0.000000, output: 0.011841
Expected: 1.000000, output: 0.984676
\end{lstlisting}

Sieć w tej konfiguracji rozpoznała poprawnie 7/13 kwiatów.

\subsubsection{Konfiguracja 4-5-4-3, 140000 epok}
\begin{lstlisting}[basicstyle=\small]
=== Attributes: 5.1 3.5 1.4 0.2 
Expected: 1.000000, output: 0.997355
Expected: 0.000000, output: 0.002485
Expected: 0.000000, output: 0.000002
=== Attributes: 4.7 3.2 1.3 0.2 
Expected: 1.000000, output: 0.997342
Expected: 0.000000, output: 0.002526
Expected: 0.000000, output: 0.000002
=== Attributes: 5.2 3.5 1.5 0.2 
Expected: 1.000000, output: 0.997350
Expected: 0.000000, output: 0.002502
Expected: 0.000000, output: 0.000002
=== Attributes: 5.1 3.8 1.9 0.4 
Expected: 1.000000, output: 0.997331
Expected: 0.000000, output: 0.002566
Expected: 0.000000, output: 0.000002
=== Attributes: 5 2 3.5 1 
Expected: 0.000000, output: 0.003549
Expected: 1.000000, output: 0.998728
Expected: 0.000000, output: 0.000485
=== Attributes: 6.8 2.8 4.8 1.4 
Expected: 0.000000, output: 0.003446
Expected: 1.000000, output: 0.998767
Expected: 0.000000, output: 0.000493
=== Attributes: 5.5 2.6 4.4 1.2 
Expected: 0.000000, output: 0.003463
Expected: 1.000000, output: 0.998760
Expected: 0.000000, output: 0.000492



=== Attributes: 6.3 3.3 6 2.5 
Expected: 0.000000, output: 0.000504
Expected: 0.000000, output: 0.000435
Expected: 1.000000, output: 0.999513
=== Attributes: 7.6 3 6.6 2.1 
Expected: 0.000000, output: 0.000504
Expected: 0.000000, output: 0.000435
Expected: 1.000000, output: 0.999513
=== Attributes: 6 2.2 5 1.5 
Expected: 0.000000, output: 0.000504
Expected: 0.000000, output: 0.000435
Expected: 1.000000, output: 0.999513
=== Attributes: 5.9 3 5.1 1.8 
Expected: 0.000000, output: 0.000504
Expected: 0.000000, output: 0.000435
Expected: 1.000000, output: 0.999513
\end{lstlisting}

Sieć w tej konfiguracji rozpoznała poprawnie wszystkie kwiaty.
\subsection{Konfiguracja 4-4-4-4-3, 10000 epok}
\begin{lstlisting}[basicstyle=\small]
=== Attributes: 4.9 3 1.4 0.2 
Expected: 1.000000, output: 0.289672
Expected: 0.000000, output: 0.387226
Expected: 0.000000, output: 0.322644
=== Attributes: 5 3.6 1.4 0.2 
Expected: 1.000000, output: 0.289696
Expected: 0.000000, output: 0.387230
Expected: 0.000000, output: 0.322672
=== Attributes: 5.7 4.4 1.5 0.4 
Expected: 1.000000, output: 0.289730
Expected: 0.000000, output: 0.387237
Expected: 0.000000, output: 0.322712
=== Attributes: 5.1 3.8 1.5 0.3 
Expected: 1.000000, output: 0.289707
Expected: 0.000000, output: 0.387232
Expected: 0.000000, output: 0.322685
=== Attributes: 5.1 3.7 1.5 0.4 
Expected: 1.000000, output: 0.289706
Expected: 0.000000, output: 0.387232
Expected: 0.000000, output: 0.322683
=== Attributes: 4.7 3.2 1.6 0.2 
Expected: 1.000000, output: 0.289672
Expected: 0.000000, output: 0.387226
Expected: 0.000000, output: 0.322644
=== Attributes: 4.9 3.1 1.5 0.1 
Expected: 1.000000, output: 0.289678
Expected: 0.000000, output: 0.387227
Expected: 0.000000, output: 0.322651
=== Attributes: 4.4 3 1.3 0.2 
Expected: 1.000000, output: 0.289625
Expected: 0.000000, output: 0.387216
Expected: 0.000000, output: 0.322590
=== Attributes: 6.1 3 4.6 1.4 
Expected: 0.000000, output: 0.289739
Expected: 1.000000, output: 0.387239
Expected: 0.000000, output: 0.322722
=== Attributes: 5 2.3 3.3 1 
Expected: 0.000000, output: 0.289698
Expected: 1.000000, output: 0.387234
Expected: 0.000000, output: 0.322673
=== Attributes: 6.3 3.3 6 2.5 
Expected: 0.000000, output: 0.289744
Expected: 0.000000, output: 0.387240
Expected: 1.000000, output: 0.322728
=== Attributes: 6.3 3.4 5.6 2.4 
Expected: 0.000000, output: 0.289743
Expected: 0.000000, output: 0.387240
Expected: 1.000000, output: 0.322727
=== Attributes: 6.9 3.1 5.4 2.1 
Expected: 0.000000, output: 0.289745
Expected: 0.000000, output: 0.387240
Expected: 1.000000, output: 0.322729
=== Attributes: 6.5 3 5.2 2 
Expected: 0.000000, output: 0.289743
Expected: 0.000000, output: 0.387240
Expected: 1.000000, output: 0.322727
\end{lstlisting}

Sieć w tej konfiguracji ani razu nie zbliżyła się do poprawnej odpowiedzi.
\subsubsection{Konfiguracja 4-4-4-4-3, 140000 epok}
\begin{lstlisting}[basicstyle=\small]
=== Attributes: 5.1 3.5 1.4 0.2 
Expected: 1.000000, output: 0.997355
Expected: 0.000000, output: 0.002485
Expected: 0.000000, output: 0.000002
=== Attributes: 4.7 3.2 1.3 0.2 
Expected: 1.000000, output: 0.997342
Expected: 0.000000, output: 0.002526
Expected: 0.000000, output: 0.000002
=== Attributes: 5.2 3.5 1.5 0.2 
Expected: 1.000000, output: 0.997350
Expected: 0.000000, output: 0.002502
Expected: 0.000000, output: 0.000002
=== Attributes: 5.1 3.8 1.9 0.4 
Expected: 1.000000, output: 0.997331
Expected: 0.000000, output: 0.002566
Expected: 0.000000, output: 0.000002
=== Attributes: 5 2 3.5 1 
Expected: 0.000000, output: 0.003549
Expected: 1.000000, output: 0.998728
Expected: 0.000000, output: 0.000485
=== Attributes: 6.8 2.8 4.8 1.4 
Expected: 0.000000, output: 0.003446
Expected: 1.000000, output: 0.998767
Expected: 0.000000, output: 0.000493
=== Attributes: 5.5 2.6 4.4 1.2 
Expected: 0.000000, output: 0.003463
Expected: 1.000000, output: 0.998760
Expected: 0.000000, output: 0.000492



=== Attributes: 6.3 3.3 6 2.5 
Expected: 0.000000, output: 0.000504
Expected: 0.000000, output: 0.000435
Expected: 1.000000, output: 0.999513
=== Attributes: 7.6 3 6.6 2.1 
Expected: 0.000000, output: 0.000504
Expected: 0.000000, output: 0.000435
Expected: 1.000000, output: 0.999513
=== Attributes: 6 2.2 5 1.5 
Expected: 0.000000, output: 0.000504
Expected: 0.000000, output: 0.000435
Expected: 1.000000, output: 0.999513
=== Attributes: 5.9 3 5.1 1.8 
Expected: 0.000000, output: 0.000504
Expected: 0.000000, output: 0.000435
Expected: 1.000000, output: 0.999513
\end{lstlisting}

Sieć w tej Konfiguracji rozpoznała wszystkie kwiaty.

\section{Wnioski}
Perceptron wielowarstwowy może z powodzeniem służyć jako klasyfikator irysów, do realizacji tego zadania potrzeba skonfigurować sieć z przynajmniej jedną warstwą ukrytą. Proces bezbłędnej klasyfikacji przebiegł najszybciej dla modelu "4-7-3/10000".


Można zauważyć ciekawą zależność - Irys o atrybutach 6.3 2.5 4.9 1.5 stanowił problem dla sieci trenowanych przez 140000 iteracji. Ponadto, najbliżej poprawnej klasyfikacji była sieć "4-3-3/10000".
\begin{thebibliography}{0}
  \bibitem{l2short}
    \textsl{http://archive.ics.uci.edu/ml/datasets/Iris}
\end{thebibliography}
\end{document}
